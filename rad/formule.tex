\documentclass{article}

\usepackage[T2A]{fontenc}
\usepackage[utf8]{inputenc}
\usepackage[serbian]{babel}

\usepackage{hyperref}
\usepackage[
a4paper,
top=2.5cm,
left=2.5cm,
bottom=2.5cm,
right=2.5cm
]{geometry}

\title{Bitka za Ivo Džimu - formule}
\author{}
\date{\today}

% Za bolje razlomnke?
\newcommand\ddfrac[2]{\frac{\displaystyle #1}{\displaystyle #2}}
% Alias radi lakseg pisanja
\newcommand{\laj}{\sqrt{\lambda_J\lambda_A}}
% Da ne uvlaci svaki novi red
\setlength\parindent{0pt}

\begin{document}
\maketitle
%\tableofcontents


\section*{Uvod}

\(A_0 = 54000\), efikasnost \(\lambda_A = 0.0106\) \\
\(J_0 = 21500\), efikasnost \(\lambda_J = 0.0544\) \\

\(\frac{dA}{dt} = - \lambda_{J}*J\) \\
\(\frac{dJ}{dt} = - \lambda_{A}A\) \\

Resenje moze da se dobije tako sto uklonimo $t$ deljenjem jedacina:

\[
  \frac{\displaystyle \frac{\displaystyle dA}{\displaystyle dt}}{\displaystyle  \frac{\displaystyle dJ}{\displaystyle dt} } = \frac{\displaystyle dA}{\displaystyle dJ}
\]

\[
  \frac{\displaystyle dA}{\displaystyle dJ} = \frac{\displaystyle \lambda_{J}J}{\displaystyle \lambda_{A}A}
\]

\[
  \lambda_{A}A\,dA = \lambda_{J}J\,dJ
\]

integralimo jednacinu:

\[
  \lambda_{A}\int{} A\,dA = \lambda_{J} \int{} J\,dJ
\]

\[
  \lambda_{A} \frac{\displaystyle A^2}{\displaystyle 2} + \underbrace{c\lambda_{A}}_{c_1}
  = \lambda_{J} \frac{J^2}{2} + \underbrace{c\lambda_{J}}_{c_2}
\]

\[
  \lambda_{A} \frac{A^2}{2} - \lambda_{J} \frac{J^2}{2} = c_2 - c_1 = c
\]

uzmemo da je \(t = t_0 = 0\), onda \(A(t_0) = A_0\) i \(J(t_0) = J_0\),
dobijamo:

\[
  \lambda_{A}\frac{A_{0}^2}{2} - \lambda_{J}\frac{J_{0}^2}{2} = c
\]

, spojimo ovu jednacinu i jednacinu ?? da bi dobili

\[
  \lambda_{A}\frac{A^2}{2} - \lambda_{J}\frac{J^2}{2} =
\lambda_{A}\frac{A_{0}^2}{2} - \lambda_{J}\frac{J_{0}^2}{2}
\]

kad se sredi, jednacina izgleda ovako:

\[
  \lambda_{A}A^2 - \lambda_{A}A_{0}^2 = \lambda_{J}J^2 - \lambda{J}J_{0}^2
\]

Bitka traje dok jedna od strana ne izgubi sve vojnike. Ako je broj
Japanca na kraju 0 (\(J = 0\)), sredjivanjem (gornje) jednacine ?? se
dobija da je broj Americkih vojnika na kraju bitke jednak:

\[
  \lambda_{A}A^2 - \lambda_{A}A_{0}^2 = -\lambda_{J}J_{0}^2,
\]

\[
  \lambda_{A}A^2 = \lambda_{A}A_0^2 - \lambda_{J}J_0^2
\]

\[
  A^2 = A_{0}^2 - \frac{\lambda_J}{\lambda_A}J_0^2,
\]

\[
  A = \sqrt{A_{0}^2 - \frac{\lambda_J}{\lambda_A}J_0^2}
\]

I to vazi samo pri uslovu da je
\(\lambda_{A}A_0^2 \geq \lambda_{J}J_0^2\) (br. formule ??). Analogno se
izvodi kad je broj Amerikanaca na kraju 0 (\(A = 0\)):

\[
  J = \sqrt{J_0^2 - \frac{\lambda_A}{\lambda_J}A_0^2},\ \lambda_{A}A_0^2 \leq
\lambda_{J}J_0^2
\]

Te uslove mozemo da koristimo kako bi na pocetku videli koja je vojska
snaznija.

\(\newline\) \(\newline\) \(\newline\)

Resenje pocetnog sistema mozemo da dobijemo i preko \(t\), tako sto cemo
jednacinu (1) diferencirati po \(t\):

\[
  A'' = -\lambda_{J}J'
\]

Izrazimo \(J'\) preko (2):

\[
  A'' = -\lambda_{J}(-\lambda_{A}A) = \lambda_{A}\lambda_{J}A
\]

\[
  A'' - \lambda_A\lambda_J A = 0
\]

Dobili smo linearnu diferencijabilnu jednacinu drugog reda sa
konstantnim koeficijentima. Nju resavamo metodom karakteristicnih
funkcija: \(a^2 - \lambda_A\lambda_J = 0\), i dobijamo da je:

\[
  a_{1,2} = \pm \laj
\]

Kako su \(a_1\) i \(a_2\) realna i razlicita resenja, onda je

\[
  A(t) = c_{1}e^{t\laj} + c_2 e^{-t \laj} \ broj ??
\]

Kada gornju jednacinu diferenciramo po \(t\), dobijamo

\[
  A'(t) = c_1 \laj e^{t \laj} - c_2 \laj e^{-t \laj}
\]

Jednacine ?? i ?? cine sistem preko koga mozemo da dobijemo koeficijente
\(c_1\) i \(c_2\). Posto znamo da je \(A(0) = A_0\) i
\(A'(0) = -\lambda_{J}J_0\), resavamo sistem za \(t = 0\).

\[
  A_0 = c_1 + c_2
\]

\[
  -\lambda_{J}J_0 = c_1 \laj - c_2 \laj
\]

Pomnozimo prvu jednacinu sa \( \laj \) i dodamo je
drugoj kako bi dobili koef \( c_1 \):

\[
  A_0 \laj - \lambda_{J}J_0 = 2 c_1 \laj
\]

\[
  c_1 = \frac{A_0 \laj - \lambda_J J_0}{2\laj }
\]

Zamenom \(c_1\) u prvu jednacinu dobija se \(c_2\):

\[
  c_2 = \frac{A_0 \laj + \lambda_J J_0}{2\laj }
\]

Zamenom u jednacinu ?? dobijamo izvedeno \(A(t)\):

\[
  A(t) = \frac{A_0 \laj - \lambda_J J_0}{2 \laj } e^{t \laj } + \frac{A_0 \laj +
\lambda_J J_0}{2 \laj } e^{-t \laj}
\]

Analognim izvodjenjem se dobija \(J(t)\):

\[
  J(t) = c_1 e^{t \laj} + c_2 e^{-t \laj}
\]

\[
 c_1 = \frac{J_0 \laj- \lambda_A A_0}{2\laj},\ 
 c_2 = \frac{J_0 \laj+ \lambda_A A_0}{2 \laj}
\]

(oznaci ove koeficijente i jednacinu)

\section*{Bitka do istrebljenja}

\textbf{Koliko je dugo trajala bitka do istrebljenja jednog od
učesnika? Koliko je preostalo vojnika na pobedničkoj
strani?} \\

Zbog uslova ?? mozemo da zakljucimo da ce Japanci izgubiti bitku.
\emph{dodaj racun} \\
Kako je \(J(t) = 0\) mozemo da izrazimo \(t\) koristeci
jednacinu ??.

\[
  0 = c_1 e^{t \laj} + c_2 e^{-t \laj}
\]

\[
  -c_1 e^{t \laj} = c_2 e^{-t \laj}
\]

celu jednacinu mozemo da pomnozimo sa \(ln\) i posle malo sredjivanja se
dobije:

\[
  ln(-c_1) + t\laj= ln(c_2) - t\laj
\]

\[
  t = \frac{ln(c_2) - ln(-c_1)}{2\laj}, 
\]

gde su \(c_1\) i \(c_2\) koeficijenti vezani za jednacinu ?? Kad se
unesu konkretne vrednosti dobija se da je vreme \(t = ... \approx ...\).
Broj americkih vojnika mozemo da izracunamo po formuli ?? i to je
\(...\).

\emph{Dodati sliku ovde}

\section*{Pojacanje posle 30 dana}

\textbf{Posle 30 dana, koliko pojačanje bi trebalo da stigne Japancima da ne bi
izgubili bitku?} \\

Koliko vojnika ostane posle 30 dana mozemo egzaktno izracunati koristeci
formule ?? i ??, ubacivanjem \(t = 30\) u jednacine. Preko uslova ??
mozemo izracunati koliko najmanje Japanca treba da bi oni pobedeili
Amerikance. Naime, kako znamo da uslov ?? vazi ako su Japanci snazniji,
odatle mozemo da izvucemo koliko je novo \(J_0\).

\[
  \lambda_{A}A_0^2 \leq \lambda_{J}J_0^2
\]

\[
  J_0^2 \geq \frac{\lambda_A}{\lambda_J} A_0^2
\]

\[
  J_0 \geq \sqrt{\frac{\lambda_A}{\lambda_J} A_0^2}
\]

Zamenom brojeva u formulu i dodatkom jedinice dobija se \ldots{} pocetni
broj japanaca za pobedu (\(J_0\)).

Ako je novo \(A_0 = A(30)\) i novo \(J_0 = J(30) + \Delta J\), odatle
mozemo da izrazimo \(\Delta J = J_0 - J(30) = ...\) koje predstavlja
pojacanje Japanaca.

\emph{Dodati slike}

\section*{Kraj bitke za 28 dana}

\textbf{Ukoliko je potrebno da se pobeda ostvari za 28 dana, koliko
vojnika je neophodno da Amerikanci imaju u
početku?} \\

Posto ce Amerikanci pobediti, onda ce Japanca biti nula posle 28 dana,
tacnije \(J(28) = 0\). Potrebno je pronaci \(A_0\) koje ce biti dovoljno
za to. Koristicemo jednacinu ?? (onu J(t)), gde je \(t = 28\).

\[
0 = \frac{J_0 \laj - \lambda_A A_0}{2\laj} e^{28 \laj } +
    \frac{J_0 \laj + \lambda_A A_0}{2 \laj } e^{-28 \laj },
\]

Skratimo imenioce i pomnozimo:

\[
  0 = J_0 \laj e^{28 \laj } - \lambda_A A_0 e^{28 \laj} +
  J_0 \laj e^{-28 \laj} - \lambda_A A_0 e^{-28 \laj}
\]

\[
  A_0 \lambda_A (e^{28 \laj} - e^{-28 \laj})
=
  J_0 \laj(e^{28 \laj} + e^{-28 \laj})
\]

\[
  A_0 =
  \frac{J_0 \laj(e^{28 \laj} + e^{-28 \laj})}
  {\lambda_A (e^{28 \laj} - e^{-28 \laj})}
\]

Zamenom konkretnih vrednosti dobije se da pocetno broj americkih vojnika
jednak \(A_0 = ...\) \\

\emph{dodaj sliku}

\end{document}
